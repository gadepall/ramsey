\renewcommand{\theequation}{\theenumi}
\begin{enumerate}[label=\arabic*.,ref=\thesubsection.\theenumi]
\numberwithin{equation}{enumi}
\item In general, Fig. \ref{fig:parab} was generated using an {\em affine transformation}.

\item Express 
\begin{align}
y_2 = y_1^2
\label{eq:parab}
\end{align}
as a matrix equation.
\\
\solution  \eqref{eq:parab} can be expressed as
\begin{align}
\vec{y}^T\vec{D}\vec{y}+2\vec{g}^T\vec{y} = 0
\label{eq:parab_mat}
\end{align}
%
where 
\begin{align}
\vec{D} = \myvec{1 & 0 \\ 0 & 0} 
,
\vec{g} &= -\frac{1}{2}\myvec{0 \\ 1}
\label{eq:parab_coeffs}
\end{align}
%
\item Given 
\begin{align}
\vec{x}^T\vec{V}\vec{x}+2\vec{u}^T\vec{x}+ F = 0,
\label{eq:parab_gen}
\end{align}
where 
\begin{align}
\vec{V}=\vec{V}^T, \det(\vec{V}) = 0,
\label{eq:parab_vcond}
\end{align}
%
and $\vec{P}, \vec{c}$ such that
\begin{align}
\vec{x} = \vec{P}\vec{y}+\vec{c}.
\label{eq:parab_affine}
\end{align}
\eqref{eq:parab_affine} is known as an affine transformation.
Show that
\begin{align}
\begin{split}
\vec{D} &= \vec{P}^T\vec{V}\vec{P}
\\
\vec{g} &= \vec{P}^T\brak{\vec{V}\vec{c}+\vec{u}}
\\
F+ \vec{c}^T\vec{V}\vec{c} + 2\vec{u}^T\vec{c}&= 0
\end{split}
\label{eq:parab_parmas}
\end{align}

\solution Substituting \eqref{eq:parab_affine} in \eqref{eq:parab_gen},
\begin{align}
\brak{\vec{P}\vec{y}+\vec{c}}^T\vec{V}\brak{\vec{P}\vec{y}+\vec{c}}+2\vec{u}^T\brak{\vec{P}\vec{y}+\vec{c}}+ F = 0, 
\end{align}
which can be expressed as
\begin{multline}
\implies \vec{y}^T\vec{P}^T\vec{V}\vec{P}\vec{y}+2\brak{\vec{V}\vec{c}+\vec{u}}^T\vec{P}\vec{y}
\\
+ F+ \vec{c}^T\vec{V}\vec{c} + 2\vec{u}^T\vec{c} = 0
\label{eq:parab_simp}
\end{multline}
%
Comparing \eqref{eq:parab_simp} with \eqref{eq:parab_mat} \eqref{eq:parab_parmas} is obtained.
%
\item Show that there exists a $\vec{P}$ such that 
\begin{align}
\vec{P}^T\vec{P} = \vec{I}
\end{align}
%
Find $\vec{P}$ using
\begin{align}
\vec{D} = \vec{P}^T\vec{V}\vec{P}
\end{align}
\item Find $\vec{c}$ from \eqref{eq:parab_parmas}.
\\
\solution 
\begin{align}
\because \vec{g} &= \vec{P}^T\brak{\vec{V}\vec{c}+\vec{u}},
\\
\vec{V}\vec{c}&= \vec{P}\vec{g} - \vec{u}
\\
\implies \vec{c}^T\vec{V}\vec{c} &= \vec{c}^T\brak{\vec{P}\vec{g} - \vec{u}} = -F- 2\vec{u}^T\vec{c}
\end{align}
%
resulting in the matrix equation
\begin{align}
\myvec{\vec{V}\\ \brak{\vec{P}\vec{g} + \vec{u}}^T}\vec{c}&= \myvec{\vec{P}\vec{g} - \vec{u}\\ -F}
\end{align}
%
for computing $\vec{c}$.
\end{enumerate}
.
